\chapter*{\abstractname}
Google's Android has grown into one of the most popular mobile operating systems on the market also due to the rapidly increasing smartphone usage
over the past few years. So called ``Apps'' can be installed through
online stores, the most popular ``Google Play Store'' is being installed by default. Those Apps are being used for all kind of every day problems and
even for online banking and other sensitive tasks. Therefore they need
to be secured in terms of copy protection resistance as well as reverse
engineering capabilities. Static and dynamic obfuscation is a common
technique to provide some barriers for attackers. Google's concept of
distributing Apps through machine independent code (Dalvik Executables, DEX) and using Just-In-Time (JIT) compiling simplifies the reverse engineering of Apps massively and opens an enormous gate for patching/repackaging. Android's runtime architecture changes very frequently and recently from a Dalvik Virtual Machine JIT concept to a new Android Runtime (ART) using Ahead-Of-Time (AOT) compiling. Changes under the hood concerning DEX and its optimized native code version (OAT) are being investigated with regard to its copy protection domain.
A great part deals with dynamic code loading using JNI possibilities and its
corresponding copy protection applications. Trusted Execution Environments
are a great concept that do create a trusted world which is not accessible
by most developers and therefore not extraordinary useful for common copy
protection techniques. The possibility of distributing native code like
being routine in desktop environments will also be analyzed. A few application
examples are given regarding dynamic code loading, natively as well as in the Java world addressing concrete solutions like string encryption and license
improving. The developed concepts are meant to provide a general understanding of the described copy protection approaches.
