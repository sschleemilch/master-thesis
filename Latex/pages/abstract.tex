\chapter*{\abstractname}
Google's Android has grown into one of the most popular mobile operating systems on the market also due to the rapidly increasing smartphone usage
over the past few years. So called ``Apps'' can be installed through
online stores; the most popular ``Google Play Store'' is being installed by default on most devices.
Those Apps are being used for all kind of every day problems and
even for online banking and other sensitive tasks.
In order to protect intellectual property of used algorithms or to
prevent insertion of malicious code, these apps need to be secured
in terms of copy protection resistance as well as reverse
engineering capabilities.

Static and dynamic obfuscation is a common
technique to provide some barriers for attackers. Google's concept of
distributing Apps through machine independent code (Dalvik Executables, DEX) and using Just-In-Time (JIT) compiling, simplifies the reverse engineering of Apps massively and opens an enormous gate for patching and repackaging.

This master's thesis presents new concepts in order to avoid App copying,
patching and reverse engineering.

Android's runtime architecture changes very frequently and recently from a Dalvik Virtual Machine JIT concept to a new Android Runtime (ART) using Ahead-Of-Time (AOT) compiling. Those changes are affecting the use of DEX and its optimized native code version (OAT). Therefore, they are investigated with regard to their copy protection domain.

A great part deals with dynamic code loading using JNI possibilities and its
corresponding copy protection applications. It complicates the reverse code
engineering of Apps enormously, but cannot avoid it completely.
Another concept are Trusted Execution Environments
that do create a trusted world. Unfortunately due to a high fee, it is not accessible for most developers and therefore not extraordinary useful for common copy protection techniques.

The possibility of distributing native code, like being routine in desktop environments, will also be analyzed.

This master's thesis provides application examples regarding dynamic code loading, natively as well as in the Java world. They address concrete solutions like string encryption and licensing improving.
Since there is an AOT compiling step, the performance is nearly
independent from its implemented language.

Developed concepts in this master's thesis provide a general understanding of different copy protection approaches in conjunction with ART.
