\begin{appendices}
\chapter{NDK Project}\label{chapter:ndk_sample_project}

\autoref{fig:ndk_proj_tree} shows the interesting part of the Android Studio project tree
with parts that has to be changed in order to make it work.

\begin{figure}[htb]
  \dirtree{%
.1 /.
.2 app/.
.3 src/.
.4 main/.
.5 assets/.
.5 java/.
.6 package.name/.
.7 MainActivity.java.
.7 MyNDK.java.
.5 jni/.
.6 Android.mk.
.6 Application.mk.
.6 mylib.cpp.
.5 libs/.
.6 arm64-v8a/.
.7 libmylib.so.
.6 armeabi/.
.7 libmylib.so.
.6 armeabi-v7a/.
.7 libmylib.so.
.6 mips/.
.7 libmylib.so.
.6 mips64/.
.7 libmylib.so.
.6 x86/.
.7 libmylib.so.
.6 x86\_64/.
.7 libmylib.so.
.3 build.gradle.
.2 gradle.properties.
}
\caption[NDK Project Tree Cutout]{NDK Project Tree Cutout}
\label{fig:ndk_proj_tree}
\end{figure}

\begin{lstlisting}[language=Java, caption=MainActivity.java, label=nkd_sample_main]

package ma.schleemilch.nativestuff;

import android.content.Context;
import android.support.v7.app.AppCompatActivity;
import android.os.Bundle;
import android.util.Log;

public class MainActivity extends AppCompatActivity {

    public static String TAG = "MYLOG";

    @Override
    protected void onCreate(Bundle savedInstanceState) {
        super.onCreate(savedInstanceState);
        setContentView(R.layout.activity_main);

        MyNDK ndk = new MyNDK();
        ndk.showNativeMessage("Success");
    }
}
\end{lstlisting}

\begin{lstlisting}[language=Java, caption=MyNDK.java, label=nkd_sample_interface]
package ma.schleemilch.nativestuff;

public class MyNDK {
    static {
        System.loadLibrary("MyLib");
    }
    public native void showNativeMessage(String msg);
}
\end{lstlisting}

\begin{lstlisting}[language=C++, caption=mylib.cpp, label=nkd_sample_cpp]
#include "ma_schleemilch_nativestuff_MyNDK.h"
#include <string.h>

#include <android/log.h>

#define LOG_TAG "MYLOG"

#define LOGD(...) __android_log_print(ANDROID_LOG_DEBUG, LOG_TAG, __VA_ARGS__)
#define LOGE(...) __android_log_print(ANDROID_LOG_ERROR, LOG_TAG, __VA_ARGS__)

JNIEXPORT void JNICALL Java_ma_schleemilch_nativestuff_MyNDK_libExe
        (JNIEnv * env, jobject jobj, jstring msg){
        	const char *msg = env->GetStringUTFChars(msg, NULL);
        LOGD("My native message: %s", msg);
}
\end{lstlisting}

\begin{lstlisting}[language=make, caption=Android.mk, label=nkd_sample_make]
LOCAL_PATH := $(call my-dir)

include $(CLEAR_VARS)

LOCAL_MODULE := mylib
LOCAL_SRC_FILES := mylib.cpp
LOCAL_LDLIBS := -llog
include $(BUILD_SHARED_LIBRARY)
\end{lstlisting}

\begin{lstlisting}[language=make, caption=Application.mk, label=nkd_sample_app_make]
APP_MODULES := mylib
APP_ABI := all
\end{lstlisting}

Just showing the adapted part inside of \code{defaultConfig\{\ldots\}}:
\begin{lstlisting}[language=xml, caption=build.gradle, label=nkd_sample_gradle]
ndk {
	moduleName "schleemilch"
}
sourceSets.main {
	jni.srcDirs = []
	jniLibs.srcDir "src/main/libs"
}
\end{lstlisting}


\chapter{NDK AES Implementation}\label{chapter:ndk_aes_implementation}
\begin{lstlisting}[language=C++, caption=AES Encrypt(), label=nkd_aes_encrypt]
#include "openssl/aes.h"
uint8_t iv[16] = { 0x01, 0x02, 0x03, 0x04, 0x05, 0x06, 0x07, 0x08,
                   0x09, 0x0A, 0x0B, 0x0C, 0x0D, 0x0E, 0x0F, 0x10};
uint8_t inputslength;

JNIEXPORT jbyteArray JNICALL Java_schleemilch_ma_nativememory_MyNDK_encrypt (JNIEnv *env, jobject obj, jstring str, jstring jkey){
    const char *input = env->GetStringUTFChars(str, NULL);
    const char *tkey = env->GetStringUTFChars(jkey, NULL);
    int keylength = env->GetStringLength(jkey)*8;
    uint8_t key[keylength/8];
    memcpy(key, tkey, keylength/8);

    uint8_t aes_key[keylength/8];
    memset(aes_key, 0, keylength/8);
    inputslength = env->GetStringLength(str);
    uint8_t aes_input[inputslength];
    memcpy(aes_input,input,inputslength);

    uint8_t iv_enc[AES_BLOCK_SIZE];
    memcpy(iv_enc,iv,AES_BLOCK_SIZE);

    const size_t encslength = ((inputslength + AES_BLOCK_SIZE) / AES_BLOCK_SIZE) * AES_BLOCK_SIZE;
    unsigned char enc_out[encslength];
    memset(enc_out, 0, sizeof(enc_out));

    AES_KEY enc_key, dec_key;
    AES_set_encrypt_key(aes_key, keylength, &enc_key);
    AES_cbc_encrypt(aes_input, enc_out, inputslength, &enc_key, iv_enc, AES_ENCRYPT);

    jbyteArray ret = env->NewByteArray(AES_BLOCK_SIZE);
    env->SetByteArrayRegion(ret,0,AES_BLOCK_SIZE, reinterpret_cast<jbyte *>(enc_out));
    return ret;
}
\end{lstlisting}
\begin{lstlisting}[language=C++, caption=AES Decrypt(), label=nkd_aes_decrypt]
JNIEXPORT jbyteArray JNICALL Java_schleemilch_ma_nativememory_MyNDK_decrypt (JNIEnv *env, jobject obj, jbyteArray jencrypted, jstring jkey){
    const char *tkey = env->GetStringUTFChars(jkey, NULL);
    int keylength = env->GetStringLength(jkey)*8;
    uint8_t key[keylength/8];
    memcpy(key, tkey, keylength/8);

    uint8_t aes_key[keylength/8];
    memset(aes_key, 0, keylength/8);

    uint8_t iv_dec[AES_BLOCK_SIZE];
    memcpy(iv_dec,iv,AES_BLOCK_SIZE);

    unsigned char dec_out[inputslength];
    memset(dec_out, 0, sizeof(dec_out));

    const size_t encslength = ((inputslength + AES_BLOCK_SIZE) / AES_BLOCK_SIZE) * AES_BLOCK_SIZE;
    unsigned char enc_out[env->GetArrayLength(jencrypted)];
    env->GetByteArrayRegion(jencrypted,0, sizeof(enc_out), reinterpret_cast<jbyte*>(enc_out));

    AES_KEY dec_key;
    AES_set_decrypt_key(aes_key, keylength, &dec_key);
    AES_cbc_encrypt(enc_out, dec_out, encslength, &dec_key, iv_dec, AES_DECRYPT);

    jbyteArray ret = env->NewByteArray(sizeof(dec_out));
    env->SetByteArrayRegion(ret,0, sizeof(dec_out), reinterpret_cast<jbyte *>(dec_out));
    return ret;
}
\end{lstlisting}

\begin{figure}[htb]
  \dirtree{%
.1 jni/.
.2 Memory/.
.3 armeabi-v7a/.
.4 lib/.
.5 libcrypto.a.
.5 libcrypto.so.
.5 libssl.a.
.5 libssl.so.
.3 openssl/.
.4 aes.h.
.4 \ldots.
.3 x86/.
.3 Android.mk.
.3 memory.cpp.
.3 schleemilch\_ma\_nativememory\_MyNDK.h.
.2 Android.mk.
.2 Application.mk.
}
\caption[NDK AES Implementation JNI Tree]{NDK AES Implementation JNI Tree}
\label{fig:ndk_aes_impl_tree}
\end{figure}
\begin{lstlisting}[language=make, caption=Memory/Android.mk, label=nkd_aes_make]
LOCAL_PATH := $(call my-dir)

include $(CLEAR_VARS)
LOCAL_MODULE := opencrypto_static
LOCAL_SRC_FILES := $(TARGET_ARCH_ABI)/lib/libcrypto.a
include $(PREBUILT_STATIC_LIBRARY)

include $(CLEAR_VARS)
LOCAL_MODULE := Memory
LOCAL_SRC_FILES := memory.cpp
LOCAL_LDLIBS := -llog
LOCAL_C_INCLUDES:= openssl
LOCAL_SHARED_LIBRARIES := opencrypto_static
include $(BUILD_SHARED_LIBRARY)
\end{lstlisting}

\begin{lstlisting}[language=C++, caption=Decompiled AES Encrypt(), label=dec_aes_encrypt]
int32_t Java_schleemilch_ma_nativememory_MyNDK_encrypt(struct struct_6 a1, int32_t a2, int32_t a3, int32_t a4, int32_t a5, int32_t a6, int32_t a7, int32_t a8, int32_t a9, int32_t a10, int32_t a11) {
    struct struct_6 v1; // 0x1560_7
    struct struct_6 v2; // 0x1594_8
    g32 = a4;
    struct struct_6 * v3;
    g34 = (int32_t)&v3;
    g35 = *(int32_t *)0x6f78;
    g33 = a3;
    g29 = 0;
    g2 = false;
    g4 = true;
    _ZN7_JNIEnv17GetStringUTFCharsEP8_jstringPh((struct struct_6 *)a1.e0, (char *)a3);
    g2 = false;
    g4 = true;
    int32_t v4 = g31; // 0x1552
    _ZN7_JNIEnv17GetStringUTFCharsEP8_jstringPh((struct struct_6 *)v4, (char *)g32);
    g26 = v4;
    int32_t v5 = g31; // 0x155e
    v1 = (struct struct_6){
        .e0 = 0,
        .e1 = 0
    };
    v1.e0 = v5;
    _ZN7_JNIEnv15GetStringLengthEP8_jstring(v1, g32, g5);
    g32 = v5;
    int32_t v6 = v5 + 7 & -8; // 0x156c
    g36 = v6;
    int32_t v7 = g37 - v6; // 0x1572
    memcpy((char *)(v7 + 8), (char *)g26, v5);
    int32_t v8 = v7 - g36; // 0x157c
    g37 = v8;
    int32_t v9 = v8 + 8; // 0x1580
    g26 = v9;
    int32_t v10 = g32; // 0x1584
    g2 = false;
    g4 = true;
    memset((char *)v9, g5, v10);
    int32_t v11 = g33; // 0x158e
    int32_t v12 = g31; // 0x1590
    g33 = 0x59e0;
    v2 = (struct struct_6){
        .e0 = 0,
        .e1 = 0
    };
    v2.e0 = v12;
    _ZN7_JNIEnv15GetStringLengthEP8_jstring(v2, v11, v10);
    int32_t v13 = g33 + 0x159c; // 0x1598
    g33 = v13;
    int32_t v14 = *(int32_t *)v13; // 0x159a
    g33 = v14;
    int32_t v15 = v12 % 256; // R11
    *(char *)v14 = (char)v12;
    int32_t v16 = g37 - (v15 + 7 & 504); // 0x15ac
    int32_t v17 = v16 + 8; // 0x15b2
    g36 = v17;
    memcpy((char *)v17, (char *)v3, v15);
    int32_t v18;
    int32_t v19 = &v18; // 0x15c0_0
    int32_t v20 = *(int32_t *)0x6f80; // 0x15ca
    g35 = v19;
    int32_t v21 = v20; // 0x15d2
    // branch -> 0x15d2
    while (true) {
        int32_t v22 = v21 + 8; // 0x15d4
        *(int32_t *)v19 = *(int32_t *)v21;
        *(int32_t *)(v19 + 4) = *(int32_t *)(v21 + 4);
        int32_t v23 = v19 + 8; // 0x15de
        g28 = v23;
        g23 = v23;
        if (v22 == v20 + 16) {
            int32_t v24 = v15 + 16 & 496; // 0x15ec
            int32_t v25 = v16 - v24; // 0x15f0
            g37 = v25;
            int32_t v26 = v25 + 8; // 0x15f4
            g27 = v26;
            memset((char *)v26, g5, v24);
            g25 = 8 * g32;
            int32_t v27;
            int32_t v28 = &v27; // 0x1600_0
            g32 = v28;
            g24 = g26;
            g29 = v28;
            AES_set_encrypt_key();
            g29 = (int32_t)*(char *)g33;
            *(int32_t *)(g37 + 4) = 1;
            g30 = v28;
            *(int32_t *)g37 = g35;
            AES_cbc_encrypt(g36, g27);
            int32_t v29 = g31; // 0x1620
            g25 = 16;
            g2 = false;
            g4 = false;
            g24 = v29;
            int32_t v30 = *(int32_t *)(*(int32_t *)v29 + 704); // 0x1626
            g30 = v30;
            g23 = 0x162d;
            ((int32_t (*)())(v30 & -2))();
            int32_t v31 = g31; // 0x162c
            g29 = 0;
            int32_t v32 = *(int32_t *)(*(int32_t *)v31 + 832); // 0x1634
            g33 = v32;
            g30 = 16;
            g2 = false;
            g4 = false;
            int32_t v33 = g24; // 0x163a
            g32 = v33;
            g24 = v31;
            g25 = v33;
            g23 = 0x1643;
            ((int32_t (*)())(v32 & -2))();
            int32_t v34 = g34; // 0x1642
            int32_t v35 = *(int32_t *)(v34 + 4); // 0x1642
            g25 = v35;
            uint32_t v36 = *(int32_t *)(v34 + 268); // 0x1644
            g29 = v36;
            g24 = g32;
            int32_t v37 = *(int32_t *)v35; // 0x164a
            g30 = v37;
            uint32_t v38 = -2 - v37 + v36; // 0x164c
            g3 = ((v38 ^ v36) & (v38 ^ -v37)) < 0;
            g2 = v36 - v37 < 0;
            g4 = v36 == v37;
            g1 = v38 <= v36;
            int32_t v39; // 0x1654
            if (v36 != v37) {
                // 0x1650
                __stack_chk_fail();
                v39 = g34;
                // branch -> 0x1654
            } else {
                v39 = v34;
            }
            // 0x1654
            g31 = *(int32_t *)(v39 + 276);
            g32 = *(int32_t *)(v39 + 280);
            g33 = *(int32_t *)(v39 + 284);
            g34 = *(int32_t *)(v39 + 288);
            g35 = *(int32_t *)(v39 + 292);
            g36 = *(int32_t *)(v39 + 296);
            g26 = *(int32_t *)(v39 + 300);
            g27 = *(int32_t *)(v39 + 304);
            g37 = v39 + 312;
            ((int32_t (*)())*(int32_t *)(v39 + 308))();
            return *(int32_t *)(g31 + g34);
        }
        // 0x15d2
        v19 = v23;
        v21 = v22;
        // branch -> 0x15d2
    }
}
\end{lstlisting}



\end{appendices}
