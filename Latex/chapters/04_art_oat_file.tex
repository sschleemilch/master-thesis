\chapter{ART App Executable}
\label{chapter:art_oat_file_inspection}

A core element of the recently introduced ART is the file that
gets created by ``dex2oat'' during the installation time of an app,
described in \autoref{section:app_installation}.
Since ART does use AOT compilation, the file format is expected
to be an executable or at least a native code container.
A lot of copy protection mechanisms are based on the use of native
code because it is supposed to be more secure from reverse code
engineering (which is an assumption so far).
Therefore it's worth to have closer look at that file format
especially since Google does not provide any further information
about it's content and it might have the potential of
revolutionizing the available copy protection mechanisms for
Android or having at least a great impact on them.

By applying the Unix command ``\code{file}'' (which can classify
files to MIME-types) to the resulting file of the ``\code{dex2oat}''
tool it comes apparent that it's a particular ELF file (32 or 64 bit)
called OAT file from now on.

\section{ELF File Format}\label{section:elf_file_format}
ELF originally was originally specified by UNIX System Laboratories
(USL) and later by Tool Interface Standards (TIS) and is a common
standard for executables, object code and shared libraries on UNIX
systems. It's a quite flexible format for different CPUs and
architectures and does serve as a container for different
executable binary formats.

\begin{figure}[htb]
  \centering
  \includegraphics[scale=0.4]{figures/elf_format}
  \caption[ELF file format]{ELF file format taken from \parencite{portable_formats_spec}}
  \label{fig:elf_format}
\end{figure}

The \autoref{fig:elf_format} shows the file structure.
One have to differentiate the context of how the file is viewed.
While a linker does care about sections, sections may be
glued together to segments when executing the file.
Meta data about the file can be read out of the ``ELF header''
that starts at adress \code{0x00} and does contain
information about the version, file type, target machine and
offsets to the program- and section header tables.
In ART, the file is marked as an shared object with LSB encoding
and not as an executable which makes clear that this file is not
supposed to get executed directly but to be linked first
(An open question remains so far which process is then starting
the app).
Segments are referenced by the program header table and sections
by the section header table. For an execution process, only the header
and the information out of the program header table is needed
\parencite{life_of_binaries}.

Let's first have a look at the used sections in case of the specific
Android implementation, the OAT file:
\autoref{fig:section_headers} does show the output of
``\code{readelf -S <ELF-App-File>}'', listening all available sections

\begin{figure}[htb]
  \centering
  \includegraphics[width=\textwidth]{figures/android_elf_section_headers}
  \caption[ELF section headers]{ELF section headers}
  \label{fig:section_headers}
\end{figure}

It does follow a short description of sections that are implemented
\parencite{life_of_binaries}:
\begin{itemize}
    \item \code{.dynsym} holds a dynamic linking symbol table that
    does contain information for locating and relocating a program's
    symbol definitions and references. It does contain ``oatdata'',
    ``oatexec'' and ``oatlastword'' in case of an OAT file.
    \item \code{.dynstr} does hold strings for dynamic linking,
    mostly names that are referenced by \code{.dynsym}.
    \item \code{.hash} contains the symbol hash table
    \item \code{.rodata} stands for ``Read-Only data'' and does
    contain arbitrary data whose interpretation is solely
    determined by the program itself. We will
    see that in case of Android it does hold the actual OAT file
    that will be further described in \autoref{section:oat_file}.
    \item \code{.text} is the only region that is marked as
    executable and therefore it does hold the main body of
    program code.
    \item \code{.dynamic} includes dynamic linking information.
    \item \code{.shstrtab} stands for ``Section header string
    table'' and therefore contains the previous described
    section names including its own (e.g. ``\code{.shstrtab}'').
\end{itemize}

\autoref{fig:program_headers} shows the alternative view of the
file by having a look at segments (``\code{readelf -S <ELF-App-File>}'')
. Type ``PHDR'' stands for ``Program header''. Segments with type
``LOAD'' are supposed to be loaded from disk into memory while
a ``DYNAMIC'' segment is a part of a ``LOAD'' segment and is equal
to the ``\code{.dynamic}'' section. The mapping of ``LOAD'' segments
into memory is performed by respecting the alignment of \code{0x1000}
means that only chunks of that size (or a multiple) are being read
(e.g. reading the segment at \code{0x3000} will copy the content
from \code{0x3000-0x4000} even if the size only equals \code{0x340}).
\begin{figure}[htb]
  \centering
  \includegraphics[width=\textwidth]{figures/android_elf_program_headers}
  \caption[ELF program headers]{ELF program headers}
  \label{fig:program_headers}
\end{figure}

The \code{readelf} tool is also capable of showing the resulting
mapping of sections to segments
(\autoref{fig:sections_segments_mapping}).

\begin{figure}[htb]
  \centering
  \includegraphics[scale=0.7]{figures/section_segment_mapping}
  \caption[ELF section segment mapping]{ELF section segment mapping}
  \label{fig:sections_segments_mapping}
\end{figure}

It's interesting to note that the Android usage of ELF for apps
is very minimalistic and does contain very few sections/segments
compared to a common program written in C/C++ (\code{helloWorld.c}
does include over 30 sections). As described before,
\code{.dynsym} does contain entries which tell us where to find
the OAT data, specifically the ``oatdata''(equals \code{.rodata})
and the ``oatexec'' (equals \code{.text})
section that will now be analyzed.

\section{OAT File}\label{section:oat_file}
Google does not provide any official documentation about the OAT
file format other than the source code itself
(\code{art/runtime/oat[\_file].h[c]}). \parencite{hiding_behind_art}
however gives a helpful introduction and an overview is given
in \autoref{fig:oat_format} that shows the content of ``oatdata''.

\begin{figure}[htb]
  \centering
  \includegraphics[scale=0.5]{figures/oat_format}
  \caption[OAT format]{OAT format}
  \label{fig:oat_format}
\end{figure}

Important attributes that the ``OAT Header'' contains are the
checksum over itself,
the instruction set (ARM, ARM64, MIPS, \ldots), the executable
offset from start of ``oatdata'' and the quantity of embedded
DEX files (that should be one and does exist for flexibility
reasons). It does follow the ``OAT Dex Header'', containing
the absolute path of the original DEX file, the checksum, the
offset to the copy of the DEX file that is embedded as well as
an offset to the ``OAT Class Headers''. ``OAT Class Headers''
do offer information about defined classes. First, the type
of class which contains how many methods in the class
are compiled to native code (``all'', ``some'' or ``none'' but
should be ``all'' in almost every case) and secondly
the offsets to the native code begin of every compiled method.

Überleitung zu DEX...

\section{DEX File Format}
\label{section:dex_file_format}
