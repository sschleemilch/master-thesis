\chapter{Introduction}\label{chapter:android_status_quo}

\section{Popularity of Android}
Android is an operating system by Google,
designed for mobile devices. Version 1.0 was released
in 2008 and the most recent version is 6.0 as this thesis
is written. With a remarkable market share of 82.8\%
it has grown into the most important mobile operating system
followed by Apple's iOS with 13.9\% and Microsoft's Windows
Phone with 2.6\%. See \parencite{marketshare} for a full list
of mobile operating system market shares. While the smartphone
market is still growing, Android did also adapt to new
rising platforms like wearables, TV's, cars and general
embedded devices. Android is built on a Linux kernel
and because of that, Android is open source and freely
available. Everyone is allowed to adopt it to it's own needs
(except Android Wear) which is another reason for it's
rising popularity.

The daily use of smartphones and therefore Android is increasing
rapidly in nearly every field of application. Right now the usages
are reaching from simply surfing the web to security sensitive tasks
like banking transactions, the organization of all kinds of tickets
and wireless payment methods.
One of the main factors of success are App Stores who provide simple
installable applications for nearly every imaginable task so far.

\section{Demand of App Protection Mechanisms}
The popularity of operating systems generally result in an increasing interaction between developers and a large variety of Apps. At the same time it attracts malware and virus developers, which try to exploit the system due to its market size and the resulting high possible illegal profit.
Although Google has established it's own App Store
``Google Play Store'' where Apps are getting inspected by Google's Bouncer,
it's possible to install Apps from so called ``unknown sources''
like websites and alternative App Stores.

There are different attack vectors that can be considered by attackers.
The main ones are summarized into three following scenarios:

\subsubsection*{Scenario 1: Piracy/Intellectual Property Theft}\label{section:scenario_1}
Something that is present in every domain where program code is being written
and can be sold is illegal copying of a whole executable program or copying and adapting its source code.
Not every Android App is available for free. So being
able to copy an App to another device and get it to run is a serious threat for developers. However, there do exist licensing mechanisms to check if a user is
authorized to use the App, if not, the App can be exited.
Another possibility for attackers would be to disassemble
the App into its source code to copy and adapt the App and releasing it on its own (for instance with a different name and layout).
This scenario is hazardous for Apps that do contain a lot of know-how in form of written code.


\subsubsection*{Scenario 2: App Patching}\label{section:scenario_2}
Many Apps these days are free of charge in their basic version and
do offer ``In-App Purchases'' for additional content or services.
An attacker could try to circumvent the licensing mechanism by ``patching'' the App. This means injecting code at runtime or trying to obtain its source code followed by adding code with the goal to bypass the implemented licensing
mechanism. The attacker can then repackage and reinstall the changed App
with extended access rights. This leads to a financial loss for developers.

\subsubsection*{Scenario 3: Malicious Code Injection}\label{section:scenario_3}
Quite similar to scenario one the attacker chooses a popular App he wants to exploit but this time the focus lies on getting sensitive data from users.
So a good example would be a banking App where users can do transactions and have to enter pins and TANS.
To be able to sniff sensitive data, one possibility is to get the App’s source code like in scenario one and injecting own code that implements the malicious functionality. After repackaging it, potential users need to be tricked into installing the App, which appears to be the official banking App. This can be done using common social engineering techniques since it should not be possible to upload the App to an official App store if the original App is present.\\
%With the same technique the attacker could also inject malicious
%arbitrary code followed by repacking and redistribution of that
%App to potential victims. Hereby, the attacker can sniff
%sensitive user data which can be exploited.

Therefore there is an urgent need of copy protection mechanisms
for Android Apps which basically means in most cases to prevent reverse code engineering of the distributed Application packages in order to hinder the repackaging of an App including malicious or generally altered code.
