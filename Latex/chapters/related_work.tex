\chapter{Related Work}\label{chapter:related_workd}
\parencite{anti_piracy} introduces an anti-piracy mechanism that is based on class
separation and dynamic loading at a Java level. The main concept is to separate the app source code into an ``Incomplete Main Application'' (IMA) and a ``Seperate Essential Class'' (SEC) that gets downloaded at first use of the app and decrypted after a authentication. For loading, common \code{DexClassLoading} is being used like shown in
\autoref{section:dynamic_code_loading}. This method is still possible after the ART transition but still has the downside of reverse code engineering the additional SEC 
element after a one time App execution and should be vulnerable versus dynamic reverse code engineering.

\parencite{grab_n_run} addresses the problem of developers implementing a unsafe variant of dynamic code loading techniques by calling common Android APIs like \code{dexClassLoader} so the result is a wrapper (\code{SecureDexClassLoader}) for dynamic code loading techniques again at Java layer that includes security checks for fetching code from an URL, storing the code in app-private directory, proof integrity and developer authenticity of the code, and finally load it. It is a useful practical tool for developers who are not very familiar with all kind of security risks when using dynamic code loading techniques. In the scope of copy protection mechanisms, it is not that revolutionary though but could of course be used when implementing a dynamic code loading mechanism on Java layer.

Since dynamic DEX loading is still possible, Android Packers might still be a 
good choice of protecting intellectual property. In \parencite{android_packer}
an introduction to Android Packers is given, carving out the difference between
packers and obfuscation as well as introducing popular packers like ApkProtect,
Bangcle and Iliami and future challenges in the packer domain.
\parencite{dexhunter} on the other side analyzes the major techniques 
used by the most common packers also addressing ART as wells as Dalvik.
Their developed novel system ``DexHunter'' is able to recover most DEX files so
their work indicates that packaging services are not that secure as they appear to be.

Regarding the protection of intellectual property, \parencite{forensic_mark}
shows a concept of inserting a forensic mark to an App that inserts buyers 
information right into the \code{classes.dex}. It also includes the technique
of verifying the app license with the mark as a foundation. This technique could
also be added to a license mechanism to increase its robustness.
\parencite{visual_exploration} proposes a visual method of analyzing Android
executable files including ART to reveal patterns. Although it focusses mainly
on finding anomalies of malicious Apps, it could also be used to classify 
common Apps and possibly gain additional information with less effort compared
to other reverse code engineering techniques. The authors are parsing the
DEX file and coloring its file structure transfering the gained information
to its corresponding binary. 

\parencite{key_storage} explores secure key storage options in Android that are needed for instance for encryption/decryption scenarios. It compares the built in
feature as well as the Bouncy Castle key storage solution. The security of the 
built in feature depends on the device and might not make use of ARM
TrustZone features. Bouncy Castle on the other hand can provide even stronger 
security guarantees.

Since the Google Play Store assumes to be nearly malware free introducing the Google Bouncer that rejects malicious Apps, \parencite{divide_and_conquer}
shows that it can be surpassed using a technique called ``Divide-and-Conquer''.





