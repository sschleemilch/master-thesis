\chapter{Android Status Quo}\label{chapter:android_status_quo}

\section{Popularity of Android}
Android is an operating system by Google,
designed for mobile devices. Version 1.0 was released
in 2008 and the most recent version is 6.0 as this thesis
is written. With a remarkable market share of 82.8\%
it has grown into the most important mobile operating system
followed by Apple's iOS with 13.9\% and Microsoft's Windows
Phone with 2.6\%. See~\parencite{marketsharea} for a full list
of mobile operating system market shares. While the smartphone
market is still growing, Android did also attach to new
rising platforms like wearables, TV's, cars and general
embedded devices. Android is built on a Linux kernel
and because of that, Android is also open source and freely
available. Everyone is allowed to adopt it to it's own need
(except Android Wear) which is another reason for it's
rising popularity.

The daily use of smartphones and therefore Android is increasing
rapidly in nearly every scope of application. Right now the usages
are reaching from simple surfing the web to security sensitive tasks
like banking transactions, the organization of all kinds of tickets
and wireless payment methods.

\section{Demand of app protection mechanisms}
Popularity of operating systems tends to result in a focus of
malware and virus developers as a reason to the resulting market
size and therefore the most possible illegal profit.
Although Google has established it's own app store
``Google Play Store'' where apps are getting inspected by Google,
it's possible to install apps from so called ``unknown sources''
like websites and alternative app stores.

There are different attack vectors attackers are considering.
The first one is intellectual theft by extracting the source
code of an app (which is quite simple out of the delivered
package of an app), adapt and rerelease it with different credentials.


Many apps these days are free of charge in their basic version and
do offer ``In-App Purchases'' for additional content or services.
An attacker could try to ``patch'' the app which means
injecting own code to circumvent the purchase mechanism.
With the same technique the attacker could also inject malicious
arbitrary code followed by repacking and redistribution of that
app to potential victims. Hereby, the attacker can sniff
sensitive user data which can be exploited.

Therefore there is an urgent need of copy protection mechanisms
for Android apps.
