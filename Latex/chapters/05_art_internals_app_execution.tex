\chapter{ART Internals: App Execution}
\label{chapter:art_internals_app_execution}

A running app can be tracked with linux tools that are capable
of showing processes like ``\code{ps}''. To investigate the app execution
we therefore have to open a root shell at the target device
(device should be rooted) with ``\code{adb shell}''
followed by an ``\code{su}'' command after getting the
device prompt (``\code{shell@flounder:/ \$}''). It will then change
to ``\code{root@flounder:/ \#}''. A ``\code{ps}'' command will display
useful information like it's ``USER'', the process id ``PID'',
the process id of its parent ``PPID'' and of course the actual name.
Interesting entries for further inspection are being displayed in
\autoref{tab:ps_entries}.

\begin{table}[htb]
  \caption[Android processes]{Android processes}
  \label{tab:ps_entries}
  \centering
  \begin{tabular}{l l l l l}
    \toprule
      USER & PID & PPID & ... & NAME \\
    \midrule
      root & 1 & 0 & ... & /init \\
      root & 211 & 1 & ... & zygote64 \\
      root & 212 & 1 & ... & zygote \\
      u0\_a137 & 10072 & 211 & ... & ma.schleemilch.helloandroid \\
      u0\_a35 & 11017 & 212 & ... & com.android.chrome \\
    \bottomrule
  \end{tabular}
\end{table}

The process ``\code{/init}'' is the first process of Android (although
it has a parent with PID ``0'' which is the process scheduler at kernel
level).
Furthermore, every user and system app has either the process
``\code{zygote}'' or ``\code{zygote64}'' as its parent depending
if the app was written for 32 or 64 bit. That makes clear that apps
are forked from the Zygote process that is in turn forked out of
``code{/init}''. Even more detailed information about processes can be
pulled out of the ``\code{/proc}'' directory. It is an interface to the
kernel and does contain a folder for every process, named after its PID
\parencite{proc}. The most attractive attribute of that folder is
``\code{exe}'' which is a symbolic link to the executable that started
the process. Since apps are a fork of Zygote, they should point
to the same executable, which they do.

\begin{table}[htb]
  \caption[Process starting executables]{Process starting executables}
  \label{tab:process_executables}
  \centering
  \begin{tabular}{l l}
    \toprule
    \multicolumn{2}{l}{root@flounder:/ \# ls -la /proc/10072/exe} \\
    ... & exe -> /system/bin/app\_process64\_original\\
    \midrule
    \multicolumn{2}{l}{root@flounder:/ \# ls -la /proc/211/exe} \\
    ... & exe -> /system/bin/app\_process64\_original\\
    \midrule
    \multicolumn{2}{l}{root@flounder:/ \# ls -la /proc/11017/exe} \\
    ... & exe -> /system/bin/app\_process32\\
    \midrule
    \multicolumn{2}{l}{root@flounder:/ \# ls -la /proc/212/exe} \\
    ... & exe -> /system/bin/app\_process32\\
    \bottomrule
  \end{tabular}
\end{table}
